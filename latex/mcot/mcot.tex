\documentclass[11pt]{article}

\usepackage[utf8]{inputenc}
\usepackage[french]{babel}

\usepackage[a4paper]{geometry}
\geometry{top=1.5cm, bottom=1.5cm, left=2cm, right=2cm}

\title{\textbf{Méthode de reconnaissance de formes : \\application à la reconnaissance de caractères}}
\author{Abel \bsc{Calluaud}}
\date{\oldstylenums{2017}}

\begin{document}

\maketitle


\section{Positionnement thématique}
\textit{Informatique pratique, Informatique, Algèbre linéaire}

\section{Mots clés}
\begin{tabular}{ll}

\textbf{français} & \textbf{english}\\
\hline
Apprentissage automatique & Machine learning \\
Mot clé & Keyword \\
Mot clé & Keyword \\
Mot clé & Keyword \\
Mot clé & Keyword \\
\end{tabular}

\section{Problématique}

\section{Objectifs}
\begin{enumerate}
\item Premier objectif
\item Deuxième objectif
\item Troisième objectif
\end{enumerate}


\section{Bibliographie commentée}
\cite{likforman2013reconnaissance}


\bibliographystyle{plain}
\bibliography{ref}

\end{document}
